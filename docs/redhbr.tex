% Options for packages loaded elsewhere
\PassOptionsToPackage{unicode}{hyperref}
\PassOptionsToPackage{hyphens}{url}
%
\documentclass[
]{book}
\usepackage{amsmath,amssymb}
\usepackage{lmodern}
\usepackage{ifxetex,ifluatex}
\ifnum 0\ifxetex 1\fi\ifluatex 1\fi=0 % if pdftex
  \usepackage[T1]{fontenc}
  \usepackage[utf8]{inputenc}
  \usepackage{textcomp} % provide euro and other symbols
\else % if luatex or xetex
  \usepackage{unicode-math}
  \defaultfontfeatures{Scale=MatchLowercase}
  \defaultfontfeatures[\rmfamily]{Ligatures=TeX,Scale=1}
\fi
% Use upquote if available, for straight quotes in verbatim environments
\IfFileExists{upquote.sty}{\usepackage{upquote}}{}
\IfFileExists{microtype.sty}{% use microtype if available
  \usepackage[]{microtype}
  \UseMicrotypeSet[protrusion]{basicmath} % disable protrusion for tt fonts
}{}
\makeatletter
\@ifundefined{KOMAClassName}{% if non-KOMA class
  \IfFileExists{parskip.sty}{%
    \usepackage{parskip}
  }{% else
    \setlength{\parindent}{0pt}
    \setlength{\parskip}{6pt plus 2pt minus 1pt}}
}{% if KOMA class
  \KOMAoptions{parskip=half}}
\makeatother
\usepackage{xcolor}
\IfFileExists{xurl.sty}{\usepackage{xurl}}{} % add URL line breaks if available
\IfFileExists{bookmark.sty}{\usepackage{bookmark}}{\usepackage{hyperref}}
\hypersetup{
  pdftitle={Repositório Digital das Humanidades (PT-BR)},
  pdfauthor={Leonardo F. Nascimento; Eric Brasil; Tarssio Barreto; Vítor Mussa; Outro; Outro},
  hidelinks,
  pdfcreator={LaTeX via pandoc}}
\urlstyle{same} % disable monospaced font for URLs
\usepackage{longtable,booktabs,array}
\usepackage{calc} % for calculating minipage widths
% Correct order of tables after \paragraph or \subparagraph
\usepackage{etoolbox}
\makeatletter
\patchcmd\longtable{\par}{\if@noskipsec\mbox{}\fi\par}{}{}
\makeatother
% Allow footnotes in longtable head/foot
\IfFileExists{footnotehyper.sty}{\usepackage{footnotehyper}}{\usepackage{footnote}}
\makesavenoteenv{longtable}
\usepackage{graphicx}
\makeatletter
\def\maxwidth{\ifdim\Gin@nat@width>\linewidth\linewidth\else\Gin@nat@width\fi}
\def\maxheight{\ifdim\Gin@nat@height>\textheight\textheight\else\Gin@nat@height\fi}
\makeatother
% Scale images if necessary, so that they will not overflow the page
% margins by default, and it is still possible to overwrite the defaults
% using explicit options in \includegraphics[width, height, ...]{}
\setkeys{Gin}{width=\maxwidth,height=\maxheight,keepaspectratio}
% Set default figure placement to htbp
\makeatletter
\def\fps@figure{htbp}
\makeatother
\setlength{\emergencystretch}{3em} % prevent overfull lines
\providecommand{\tightlist}{%
  \setlength{\itemsep}{0pt}\setlength{\parskip}{0pt}}
\setcounter{secnumdepth}{5}
\usepackage{booktabs}
\ifluatex
  \usepackage{selnolig}  % disable illegal ligatures
\fi
\usepackage[]{natbib}
\bibliographystyle{apalike}

\title{Repositório Digital das Humanidades (PT-BR)}
\author{Leonardo F. Nascimento\footnote{UFBA - Laboratório de Humanidades Digitais, \href{mailto:leofn@ufba.br}{\nolinkurl{leofn@ufba.br}}} \and Eric Brasil\footnote{UNILAB - LABHDUFBA, \href{mailto:john@example.org}{\nolinkurl{john@example.org}}} \and Tarssio Barreto\footnote{UFBA - Laboratório de Humanidades Digitais, \href{mailto:tarssioesa@gmail.com}{\nolinkurl{tarssioesa@gmail.com}}} \and Vítor Mussa\footnote{UFRJ - PPGA, \href{mailto:vmussa@gmail.com}{\nolinkurl{vmussa@gmail.com}}} \and Outro\footnote{UFRJ - PPGCS, \href{mailto:vmussa@gmail.com}{\nolinkurl{vmussa@gmail.com}}} \and Outro\footnote{UFRJ - PPGCS, \href{mailto:vmussa@gmail.com}{\nolinkurl{vmussa@gmail.com}}}}
\date{2021-04-25}

\begin{document}
\maketitle

{
\setcounter{tocdepth}{1}
\tableofcontents
}
\hypertarget{apresentauxe7uxe3o}{%
\chapter{Apresentação}\label{apresentauxe7uxe3o}}

A idéia desta obra foi reunir esforços de diferentes pesquisadores e instituições na elaboração de scripts para coletar - de modo automatizado - a produção intelectual dos principais congressos e eventos das áreas das humanidades.

Além disso, nós tivemos como objetivo mais amplo enfatizar a importância do desenvolvimento de habilidades computacionais por parte dos pesquisadores em todos os campos das humanidades.

Os scripts, as bases de dados e todos os documentos estão disponíveis e poderão ser baixados com apenas um clique. O acervo servirá para a realização de investigações sobre os mais variados aspectos e ampliar, com isso, o conhecimento sobre a produção acadêmica, científica e intelectual do Brasil das ciências humanas e sociais ao longo de décadas.

Para o lancamento, nós escolhemos o \href{https://dhcenternet.org/initiatives/day-of-dh/2021}{Dia Internacional das Humanidades Digitais} em 29/04/2021.

Ao compartilhar nas redes, pedimos que usem a hashtag \textbf{\#dayofdh21}

\begin{figure}
\centering
\includegraphics[width=0.35\textwidth,height=\textheight]{./img/dayofdh.jpg}
\caption{Símbolo do \#dayofdh21}
\end{figure}

\hypertarget{webscraping-e-ciuxeancias-sociais}{%
\chapter{Webscraping e ciências sociais}\label{webscraping-e-ciuxeancias-sociais}}

\hypertarget{por-que-automatizar}{%
\section{Por que automatizar?}\label{por-que-automatizar}}

\hypertarget{como-comeuxe7ar}{%
\section{Como começar?}\label{como-comeuxe7ar}}

\hypertarget{pruxf3s-e-contras}{%
\section{Prós e contras}\label{pruxf3s-e-contras}}

\hypertarget{pruxe9-requisitos}{%
\chapter{Pré-requisitos}\label{pruxe9-requisitos}}

\hypertarget{r-e-rstudio}{%
\section{R e Rstudio}\label{r-e-rstudio}}

\hypertarget{python}{%
\section{Python}\label{python}}

\hypertarget{anpuh}{%
\chapter{ANPUH}\label{anpuh}}

\hypertarget{o-que-uxe9-anpuh}{%
\section{O que é ANPUH?}\label{o-que-uxe9-anpuh}}

\hypertarget{script-de-raspagem}{%
\section{Script de raspagem}\label{script-de-raspagem}}

\hypertarget{dados}{%
\section{Dados}\label{dados}}

\hypertarget{anpocs}{%
\chapter{ANPOCS}\label{anpocs}}

\hypertarget{o-que-uxe9-a-anpocs}{%
\section{O que é a ANPOCS?}\label{o-que-uxe9-a-anpocs}}

\hypertarget{script-de-raspagem-1}{%
\section{Script de raspagem}\label{script-de-raspagem-1}}

\hypertarget{dados-1}{%
\section{Dados}\label{dados-1}}

\hypertarget{compos}{%
\chapter{ComPos}\label{compos}}

\hypertarget{o-que-uxe9-a-compuxf3s}{%
\section{O que é a Compós?}\label{o-que-uxe9-a-compuxf3s}}

\hypertarget{script-de-raspagem-2}{%
\section{Script de raspagem}\label{script-de-raspagem-2}}

\hypertarget{dados-2}{%
\section{Dados}\label{dados-2}}

\hypertarget{referuxeancias-bibliogruxe1ficas}{%
\chapter{Referências Bibliográficas}\label{referuxeancias-bibliogruxe1ficas}}

AAAAAAAAAAAAAAA

\hypertarget{sobre-os-autores}{%
\chapter{Sobre os autores}\label{sobre-os-autores}}

\hypertarget{leo}{%
\section{Leo}\label{leo}}

\hypertarget{eric}{%
\section{Eric}\label{eric}}

\hypertarget{vitor}{%
\section{Vitor}\label{vitor}}

\hypertarget{tarssio}{%
\section{Tarssio}\label{tarssio}}

\hypertarget{outro}{%
\section{Outro}\label{outro}}

\hypertarget{outro-1}{%
\section{Outro}\label{outro-1}}

\hypertarget{ajude-o-projeto}{%
\section{Ajude o projeto!}\label{ajude-o-projeto}}

\begin{figure}
\centering
\includegraphics[width=0.35\textwidth,height=\textheight]{C:/Users/Leonardo/Desktop/redhbr/img/btc.png}
\caption{bc1qmug7kcrw3kxklca7chy7c344d62gtc8fhqwnkw}
\end{figure}

  \bibliography{book.bib,packages.bib}

\end{document}
